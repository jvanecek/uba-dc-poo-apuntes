
\subsection{Polimorfismo}

En Smalltalk, el objeto 1 y \code{true} no es lo mismo. 1 sabe responder el mensaje “+”. As\'i como 0 no es \code{false}. \code{true} y \code{false} son polim\'orficos respecto de los mensaje \code{not}, \code{and} y \code{or}. 

\begin{table}[H]
\begin{tabular}{|l|l|l|}\hline
		& True & False \\\hline
  \code{not} 		& \code{\^{} false} & \code{\^{} true} \\
  \code{and: aBoolean}	& \code{\^{} aBoolean} & \code{\^{} self} \\
  \code{or: aBoolean} 	& \code{\^{} self} & \code{\^{} aBoolean} \\\hline
\end{tabular}
\end{table}

Hay que definir un objeto abstracto Boolean, con los metodos \code{not}, \code{and}, y \code{or}, que luego True y False van a implementar, y \code{true} y \code{false} van a ser instancias respectivas. Pero como Boolean tienen que tener una implementaci\'on propia de los m\'etodos, lo que se hace es poner \code{self subclassResponsability}, que si se ejecuta tira una excepci\'on. \\

\textbf{?`Qu\'e pasar\'ia si true y false fueran instancias de una sola clase Boolean?}

Entonces tendr\'iamos q hacer una implementaci\'on de \code{not} del estilo: 

\begin{verbatim}
not: 
  if( self ) ^ false
  ^ true
\end{verbatim}

Pero \code{if(..)} tendr\'ia que o bien convertirlo en \code{objeto mensaje}, o agregarlo al grupo de sintaxis de Smalltalk que ten\'iamos hasta ahora: 
\begin{enumerate}
\itemsep-0.3em
\item \code{objeto mensaje}
\item \code{variable := objeto}
\item \code{[ … ] mensaje}
\end{enumerate}

Lo que hace al lenguaje m\'as cerrado, y adem\'as el programador no puede ver c\'omo esta hecho. Como Smalltalk quiere mantener su sintaxis lo m\'as chica posible, para que sea mas customizable al programador, se implementa como \code{objeto mensaje}. 

\begin{verbatim}
not: 
  self ifTrue: [ ^ false ]
  ^ true
\end{verbatim}

Pero \code{ifTrue} deber\'ia ser entonces un mensaje que Boolean pueda contestar, asique se implementa: 

\begin{verbatim}
ifTrue: aClosure
  self ifTrue: ^ aClosure
  ^ nil
\end{verbatim}

Regresi\'on al infinitum! 

\textbf{Conculsi\'on: No se puede implementar \'algebra booleana con un solo objeto!}

Asique volviendo al caso de un Booleano abstracto, \code{ifTrue} queda implementado asi 

\begin{table}[H]
 \begin{tabular}{|l|l|l|}\hline
				& True & False \\\hline
    \code{not} 			& \code{\^{} false} & \code{\^{} true} \\
    \code{and: aBoolean}	& \code{\^{} aBoolean} & \code{\^{} self} \\
    \code{or: aBoolean} 	& \code{\^{} self} & \code{\^{} aBoolean} \\
    \code{ifTrue: aClosure} & \code{\^{} aClosure} & \code{\^{} nil} \\\hline
 \end{tabular}

\end{table}

Y mantener una sintaxis minimalista. 

Como los if son problematicos (deja al programador la tarea de pensar el programa, cuando lo deberian hacer los objetos), queremos sacarlos y queremos implementarlo con polimorfismo

\begin{verbatim}
cond1: ifTrue: [^ aBlock1 value. ]
cond2: ifTrue: [^ aBlock2 value. ]
cond3: ifTrue: [^ aBlock3 value. ]
self error: ‘...’
\end{verbatim}

\subsubsection{?`Como sacamos IFs, en pro de polimorfismo?} 

Hay una heur\'istica y consiste en seguir los siguientes pasos:
\begin{enumerate}
\itemsep-0.3em
\item Opcional (porque ya puede existir la jerarquia): Crear jerarqu\'ia polim\'orfica con una abstracci\'on por cada if
\item Copiar closure de cada if a cada abstracci\'on usando mensajes polim\'orficos
\item Ponerle nombre a la abstracci\'on
\item Ponerle nombre al mensaje
\item Opcional (porque si ya existe la jerarqu\'ia, pueden existir los objetos): buscar objeto polim\'orfico
\item Reemplazar if por “objPolimorfico mensaje”
\end{enumerate}

\textbf{!`IMPORTANTE!} Hay un l\'imite para sacar IFs: no tiene sentido sacar el if cuando los colaboradores que participan en la condici\'on no pertenecen al dominio del problema. Por ej, 

\begin{verbatim}
CtaBancaria>>withdrawl: anAmount
  (balance-anAmount) < 0 ifTrue: [ tirar excepcion ]
\end{verbatim}

En este caso \code{balance-anAmount} es un n\'umero, y no pertenece al dominio del mundo bancario. 