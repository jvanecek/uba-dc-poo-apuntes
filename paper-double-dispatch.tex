\subsection{Double Dispatch}

Sacado del paper \emph{Arithmetic and Double Dispatching in Smalltalk-80} de Kurt J. Hebel and Ralph E. Johnson. 

El \emph{Double-Dispatch} es un mecanismo en Smalltalk para ejecutar m\'etodos basados tanto en la clase del emisor como del receptor. Dan Ingalls lo llamaba m\'ultiple polimorfismo. 

En general, el m\'etodo double-dispatch para un mensaje \code{op} es

\begin{verbatim}
ReceiverClass>>op argument
  ^ argument opWordFromReceiverClass: self.
\end{verbatim}

Donde \code{op} generalmente es \code{+},\code{-},\code{*},\code{/} o \code{=}, y su \code{opWord} asociada es \code{sum}, \code{difference}, \code{product}, \code{quotient} o \code{equal} respectivamente. Por ejemplo, 

\begin{verbatim}
Fraction>>* aNumber
  ^ aNumber productFromFraction: self.
\end{verbatim}

Esto implica que todas las subclases de \code{Number} deben poder contestar el mensaje \code{productFromFraction}. Pero esto hace que si tenemos $n$ clases, para el mensaje \code{*} s\'olo, debamos implementar en el peor de los casos $n^2$ m\'etodos. Para reducir este n\'umero hay que aprovechar la herencia, y tratar de delegar la implementaci\'on a las clases padres. 

Hay principalmente tres categor\'ias de m\'etodos double-dispatch:

\begin{enumerate}
 \item Los que mandan mensajes a sus argumentos
 \item Los que heredan de la superclase del argumento
 \item Los \'ultimos m\'etodos que hacen realmente el trabajo. 
\end{enumerate}

Otra alternativa para implementar operaciones aritm\'eticas es \emph{coercion}, que si bien las operaciones para n\'umeros complejos y reales se puede implementar facilmente, no es as\'i para las operaciones con matrices. 