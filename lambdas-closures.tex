
\subsection{Lambdas vs. Closures vs. Full Closures}

\subsubsection{Closure}

Cuando se ejecuta un Closure, este bindea todas las variables que tiene al contexto de ejecucion que lo llamó. 

\begin{verbatim}
X >> m:
  | a |
  a:=1 
  ^ [a := a + 1]
\end{verbatim}

\begin{verbatim}
unX:= X new
aClosure := unX m
aClosure value. # 2
aClosure value. # 3
aClosure2 = unX m
aClosure2 value. # 2 # se crea un objeto a en un contexto de ejecucion diferente
\end{verbatim}

\code{[ .. ]} crea un nuevo objeto, y sabe que a \code{a} la busca en el contexto del que lo creo. 

\subsubsection{Lambda (Lisp)}

En los lamdas, las variables no se bindean al contexto de ejecuci\'on del que lo llama. Las definiciones de las variables se buscan en todos los contextos de ejecuci\'on. 

\begin{verbatim}
X >> m: 
  ^ [ a := a+1 ]
\end{verbatim}

Esto compila aunque \code{a} no este definido. 

\begin{verbatim}
aClosure value. # va a buscar en todos los contextos de ejecucion a a.
\end{verbatim}

\subsubsection{Full Closure (Smalltalk)}


En los Full Closure, no solo bindea las variables al contexto de ejecución que lo llamó, sino que el return dentro del mismo hace retornar del contexto de ejecución padre. 

\begin{verbatim}
X >> m
  aClosure := [ ^ 10 ]
  ^ aClosure value + 5

aClosure value. # 10, porque nunca se llega a ejecutar +5. 
  cuando se llamo a [ … ] salio del metodo. 
\end{verbatim}

\begin{verbatim}
X>> m 
  ^ [ ^ 10 ]

aClosure value. # tira error. porque primero sale del contexto de ejecucion 
  y despues quiere volver a salir desde dentro del [...]. 
\end{verbatim}

Si no tuvieramos Full Closure, no podr\'iamos implementar controles de flujo con mensajes. 

Si no se tienen Full Closure, la sintaxis tiene que implementar el If, exit (para salir del control de flujo). 

\subsubsection{While}
?`Se puede hacer lo mismo con el While? S\'i. Pero, ?`c\'omo? 

\code{[...]} en Smalltalk esto crea un objeto. 

\begin{verbatim}
a := 1
(a < 3) whileTrue: [ a:= a + 1 ]
\end{verbatim}

Como cuando whileTrue termina, vuelve a ejecutar (\code{a < 3}), y por lo tanto tiene que ser un Closure. Por lo tanto whileTrue se implementa en BlockClosure:

\begin{verbatim}
BlockClosure>>whileTrue: aBlock
  self value # evaluo la condicion
    ifTrue: [
      aBlock value.
      self whileTrue: aBlock
    ]
\end{verbatim}

Un lenguaje que elimina colas de la recursion se llaman tail recursives. ?`Se puede hacer aca? S\'i, porque puedo acceder a los contextos de ejecuci\'on, y sacar los closures que vaya apilando ah\'i. 

Asi que el \code{whileTrue} para que sea tail recursive hay que escribirlo as\'i

\begin{verbatim}
BlockClosure>>whileTrue: aBlock
  self value # evaluo la condicion
    ifTrue: [
      aBlock value.
      thisContext stack pop. # saco el contexto de ejecucion
      self whileTrue: aBlock
    ]
\end{verbatim}

% Aunque esto no termina andando en realidad. 

